\documentclass[a4paper,11pt]{uzreport}

%% --------- PAKIETY --------- %%
\usepackage{graphicx} % Obsługa grafiki
\usepackage{float} % Lepsze pozycjonowanie figur
\usepackage{listings} % Wstawianie kodu źródłowego
\usepackage{xcolor} % Kolorowanie składni
\usepackage{hyperref} % Włączenie obsługi hiperłączy
\hypersetup{
    colorlinks=true,       % Kolorowe linki zamiast ramek
    linkcolor=blue,        % Kolor linków wewnętrznych
    urlcolor=blue,         % Kolor linków URL
    citecolor=blue,        % Kolor linków do bibliografii
    pdfborder={0 0 0},     % Usuwa ramki wokół linków
    bookmarks=true,        % Włącza zakładki w PDF
    unicode=true           % Wsparcie dla znaków Unicode
}

%% --------- KONFIGURACJA LISTINGS --------- %%
\lstdefinestyle{arduinostyle}{
    language=C++,
    basicstyle=\ttfamily\footnotesize,
    keywordstyle=\color{blue}\bfseries,
    commentstyle=\color{green!60!black},
    stringstyle=\color{red},
    numbers=left,
    numberstyle=\tiny\color{gray},
    stepnumber=1,
    numbersep=5pt,
    backgroundcolor=\color{gray!10},
    showspaces=false,
    showstringspaces=false,
    showtabs=false,
    frame=single,
    tabsize=2,
    captionpos=b,
    breaklines=true,
    breakatwhitespace=false,
    rulecolor=\color{black}
}

%% --------- USTAWIENIA --------- %%
\author{% autor lub autorzy sprawozdania
  Naraziński Dawid\\
  \texttt{112102@stud.uz.zgora.pl} % przykładowy numer albumu
  \and
  Pawelski Oliwer\\
  \texttt{112109@stud.uz.zgora.pl} % przykładowy numer albumu
}
\group{34INF-SSI SP/B}
\type{p} % typ zajec p - projekt
\begin{document}

\class{Systemy wbudowane}
\lab{System kontroli dostępu RFID wraz z systemem logowania i systemem uprawnień}
\labnumber{SW 2025/2026}
\date{14.01.2026}
\supervisor{mgr inż. Norbert Łukaniszyn} % Dane z templatki
%% --------- TREŚĆ --------- %%
  \maketitle
\thispagestyle{empty} % Ukrywa numer strony na stronie tytułowej (ale liczy jako strona 1)

%% --------- SPIS TREŚCI --------- %%
\newpage % Przejście na drugą stronę (spis treści będzie na stronie 2)
\setcounter{page}{2} % Ustawienie numeru strony na 2 dla spisu treści
\tableofcontents % Generowanie spisu treści z hiperłączami
\newpage % Przejście na kolejną stronę po spisie treści

\section{Wstęp}
Głównym założeniem niniejszego projektu jest zaprojektowanie oraz budowa zaawansowanego systemu kontroli dostępu, który wykorzystuje technologię identyfikacji radiowej (RFID) do autoryzacji użytkowników. System umożliwia sterowanie fizycznym ryglem drzwi za pomocą kart oraz breloków przypisanych do konkretnych osób zarejestrowanych w systemie. Rozwiązanie oparto na nowoczesnym mikrokontrolerze XIAO ESP32C3, który dzięki zintegrowanemu modułowi Wi-Fi pozwala na komunikację z zewnętrzną bazą danych. Kluczowym aspektem projektu jest rozróżnianie poziomów uprawnień dla różnych grup użytkowników, takich jak pracownicy, kierownicy czy administratorzy. System oferuje pełną administrację zdalną poprzez interfejs sieciowy, co pozwala na zarządzanie dostępami bez konieczności fizycznej ingerencji w urządzenie. Logowanie zdarzeń odbywa się w czasie rzeczywistym, co umożliwia bieżący monitoring prób wejścia do zabezpieczonych pomieszczeń. Projekt posiada również walor dydaktyczny, pozwalając na zgłębienie zagadnień związanych z protokołami komunikacyjnymi (SPI), obsługą baz danych PostgreSQL oraz programowaniem nowoczesnych frameworków webowych. Cel projektu został zdefiniowany jako dostarczenie bezpiecznego, stabilnego i skalowalnego systemu, który może znaleźć zastosowanie w warunkach rzeczywistych, np. w biurach czy laboratoriach. Finalny produkt integruje warstwę sprzętową z chmurowym backendem, zapewniając pełną synchronizację danych.

\begin{figure}[H]
\centering
\includegraphics[width=1\textwidth]{zdjecia/image5.png}
\caption{Interfejs Dashboard - główny panel z podsumowaniem zasobów systemowych i statystykami dostępu}
\label{fig:system_overview}
\end{figure}

\section{Zagadnienia teoretyczne}
Technologia RFID (Radio-Frequency Identification) wykorzystuje fale radiowe do przesyłania danych między czytnikiem a tagiem (kartą/brelokiem). W projekcie zastosowano czytnik RC522, który pracuje na częstotliwości 13.56 MHz i komunikuje się z mikrokontrolerem za pomocą magistrali SPI.

\begin{figure}[H]
\centering
\includegraphics[width=1\textwidth]{zdjecia/image4.jpg}
\caption{Czytnik RFID-RC522 zamontowany na drewnianej szafce}
\label{fig:rfid_reader}
\end{figure}

Mikrokontroler XIAO ESP32C3, będący sercem systemu, oparty jest na architekturze RISC-V i oferuje wsparcie dla Wi-Fi oraz Bluetooth, co jest kluczowe dla funkcji IoT.

\subsection{Inicjalizacja systemu}
Funkcja \texttt{setup()} wykonywana jest jednorazowo przy starcie mikrokontrolera i odpowiada za konfigurację wszystkich używanych peryferiów oraz nawiązanie połączenia z siecią Wi-Fi.

\begin{lstlisting}[style=arduinostyle, caption={Inicjalizacja systemu - konfiguracja peryferiów\, połączenie Wi-Fi oraz przygotowanie czytnika RFID}]
void setup() {
    Serial.begin(9600);
    delay(1000);
    
    Serial.println("\n=== RFID Access Control - HTTPS ===\n");
    
    pinMode(SOLENOID_PIN, OUTPUT);
    digitalWrite(SOLENOID_PIN, LOW);
    
    SPI.begin(8, 9, 10, RFID_SS_PIN);
    mfrc522.PCD_Init();
    
    Serial.print("Laczenie z WiFi...");
    WiFi.begin(ssid, password);
    
    int attempts = 0;
    while (WiFi.status() != WL_CONNECTED && attempts < 30) {
        delay(500);
        Serial.print(".");
        attempts++;
    }
    
    if (WiFi.status() == WL_CONNECTED) {
        Serial.println(" OK");
        Serial.print("IP: ");
        Serial.println(WiFi.localIP());
    } else {
        Serial.println(" BLAD!");
    }
    
    client.setInsecure();
    
    Serial.println("\nSystem gotowy - przyloz karte RFID\n");
}
\end{lstlisting}

\subsection{Odczyt kart RFID}
Główna pętla programu nieustannie monitoruje obecność nowych kart RFID w zasięgu czytnika. Po wykryciu karty następuje odczyt unikalnego identyfikatora (UID) i weryfikacja uprawnień dostępu poprzez wywołanie API.

\begin{lstlisting}[style=arduinostyle, caption={Główna pętla programu - wykrywanie i odczyt kart RFID z mechanizmem anty-powtórzeń}]
void loop() {
    if (!mfrc522.PICC_IsNewCardPresent()) {
        delay(50);
        return;
    }
    
    if (!mfrc522.PICC_ReadCardSerial()) {
        delay(50);
        return;
    }
    
    getCardID(currentCardID);
    mfrc522.PICC_HaltA();
    mfrc522.PCD_StopCrypto1();
    delay(100);
    
    if (strcmp(currentCardID, lastCardID) == 0 && 
        (millis() - lastCardTime) < 2000) {
        return;
    }
    
    strcpy(lastCardID, currentCardID);
    lastCardTime = millis();
    
    Serial.print("\n[KARTA] Token: ");
    Serial.println(currentCardID);
    
    if (checkAccess(currentCardID)) {
        Serial.println("[DOSTEP] Przyznany!");
        openDoor();
    } else {
        Serial.println("[DOSTEP] Odmowiony!");
    }
    
    mfrc522.PCD_Init();
}
\end{lstlisting}

Komunikacja sieciowa odbywa się poprzez protokół HTTPS, przesyłając żądania w formacie JSON do API serwera. Backend systemu został zrealizowany w oparciu o Supabase, który dostarcza bazę danych PostgreSQL oraz mechanizmy autentykacji i kontroli ról (RBAC). Architektura bazy danych obejmuje relacyjne powiązania między użytkownikami, ich tokenami oraz uprawnieniami do konkretnych skanerów.

\begin{figure}[H]
\centering
\includegraphics[width=1\textwidth]{zdjecia/image10.png}
\caption{Diagram struktury bazy danych PostgreSQL z relacjami między tabelami (tokens, users, scanners, scanner\_access, access\_logs)}
\label{fig:database_schema}
\end{figure}

Interfejs użytkownika zbudowano w frameworku Next.js, co zapewnia szybkie renderowanie stron i responsywność. System ról obejmuje poziomy: Root, Admin oraz User, gdzie każda ranga posiada ściśle określone kompetencje edycyjne. Bezpieczeństwo fizyczne zapewnia elektrozamek (solenoid) 12V sterowany poprzez tranzystor pełniący rolę klucza elektronicznego. System uwzględnia również czasowe ograniczenia dostępu, co pozwala na nadawanie uprawnień z datą wygaśnięcia. Logowanie zdarzeń obejmuje nie tylko udane próby wejścia, ale także odmowy wraz z podaniem przyczyny (np. brak uprawnień, nieznany tag). Całość wdrożona jest na platformie Vercel, co gwarantuje wysoką dostępność panelu administracyjnego. System został zaprojektowany z myślą o skalowalności, co umożliwia obsługę wielu skanerów w ramach jednej infrastruktury sieciowej.

\section{Implementacja i usprawnienia}
Warstwa sprzętowa została zmontowana na podstawie schematu połączeń wykorzystującego piny GPIO mikrokontrolera ESP32C3 dla magistrali SPI (MISO, MOSI, SCK, SDA) oraz sygnału sterującego przekaźnikiem (GPIO2).

\begin{figure}[H]
\centering
\includegraphics[width=1\textwidth]{zdjecia/image7.png}
\caption{Schemat połączeń elektrycznych systemu w Cirkit Designer (ESP32C3, RC522, przekaźnik, elektrozamek)}
\label{fig:circuit_diagram}
\end{figure}

\begin{figure}[H]
\centering
\includegraphics[width=1\textwidth]{zdjecia/image11.jpg}
\caption{Wnętrze szafki z zamontowanymi komponentami systemu - ESP32C3, moduł przekaźnika, elektrozamek i okablowanie}
\label{fig:hardware_assembly}
\end{figure}

\subsection{Komunikacja HTTPS z API}
Mikrokontroler komunikuje się z backendem poprzez zabezpieczone połączenie HTTPS. Każde przyłożenie karty powoduje wysłanie żądania POST do endpointu \texttt{/api/v1/access} z danymi skanera i tokenu RFID.

\begin{lstlisting}[style=arduinostyle, caption={Weryfikacja dostępu - komunikacja HTTPS z API i parsowanie odpowiedzi JSON}]
bool checkAccess(char* cardID) {
    if (WiFi.status() != WL_CONNECTED) {
        Serial.println("BLAD: WiFi rozlaczone!");
        return false;
    }
    
    Serial.print("Sprawdzanie dostepu...");
    
    if (!client.connect(server, httpsPort)) {
        Serial.println(" BLAD polaczenia!");
        return false;
    }
    
    String jsonBody = "{\"scanner\":\"";
    jsonBody += scannerId;
    jsonBody += "\",\"token\":\"";
    jsonBody += cardID;
    jsonBody += "\"}";
    
    client.print("POST /api/v1/access HTTP/1.1\r\n");
    client.print("Host: ");
    client.print(server);
    client.print("\r\n");
    client.print("Content-Type: application/json\r\n");
    client.print("Content-Length: ");
    client.print(jsonBody.length());
    client.print("\r\n");
    client.print("Connection: close\r\n\r\n");
    client.print(jsonBody);
    
    unsigned long timeout = millis();
    while (!client.available()) {
        if (millis() - timeout > 10000) {
            Serial.println(" Timeout!");
            client.stop();
            return false;
        }
        delay(10);
    }
    
    bool inBody = false;
    bool granted = false;
    
    while (client.available()) {
        String line = client.readStringUntil('\n');
        
        if (line.length() <= 1) {
            inBody = true;
            continue;
        }
        
        if (inBody && line.indexOf("\"granted\":true") > -1) {
            granted = true;
        }
    }
    
    client.stop();
    Serial.println(granted ? " OK" : " ODMOWA");
    return granted;
}
\end{lstlisting}

\subsection{Sterowanie elektrozamkiem}
Po pozytywnej weryfikacji uprawnień aktywowany jest elektromagnes (solenoid), który odryglowuje zamek na czas 3 sekund, umożliwiając otwarcie drzwi.

\begin{lstlisting}[style=arduinostyle, caption={Sterowanie elektrozamkiem - aktywacja przekaźnika i opóźnienie czasowe}]
void openDoor() {
    Serial.println("Otwieranie drzwi...");
    digitalWrite(SOLENOID_PIN, HIGH);
    delay(3000);
    digitalWrite(SOLENOID_PIN, LOW);
    Serial.println("Zamknieto\n");
}
\end{lstlisting}

\begin{itemize}
    \item \textbf{Szyfrowanie HTTPS:} Zabezpieczenie komunikacji między ESP32 a API za pomocą certyfikatów SSL/TLS, co uniemożliwia podsłuchanie danych tokenów w sieci lokalnej.
    \item \textbf{Mechanizm anty-powtórzeń:} System zapobiega wielokrotnemu odczytowi tej samej karty w krótkim odstępie czasu (2 sekundy), eliminując przypadkowe wielokrotne otwarcia.
    \item \textbf{Timeout połączenia:} Implementacja limitu czasowego (10 sekund) dla żądań HTTP zabezpiecza przed zablokowaniem urządzenia przy problemach z siecią.
\end{itemize}

\subsection{Panel zarządzania webowego}
System posiada rozbudowany webowy panel administracyjny z kontrolą dostępu opartą na rolach użytkowników (RBAC - Role-Based Access Control). Interfejs umożliwia kompleksowe zarządzanie wszystkimi aspektami systemu kontroli dostępu przez przeglądarkę internetową. Strona publiczna \texttt{/login} służy do logowania do systemu za pomocą email i hasła. Dostęp do panelu administracyjnego wymaga posiadania roli \textbf{root} lub \textbf{admin}, przy czym zakres uprawnień różni się w zależności od poziomu dostępu.

\begin{table}[!ht]
\centering
\footnotesize
\begin{tabular}{|l|l|p{5cm}|p{4.5cm}|}
\hline
\textbf{Ścieżka} & \textbf{Moduł} & \textbf{Root} & \textbf{Admin} \\ \hline
\texttt{/dashboard} & Strona główna & \multicolumn{2}{c|}{Statystyki, wykresy, logi, akcje} \\ \hline
\texttt{/dashboard/users} & Użytkownicy & CRUD wszystkie role, reset haseł & CRUD tylko rola 'user', reset haseł \\ \hline
\texttt{/dashboard/scanners} & Skanery RFID & CRUD, konfiguracja lokalizacji & Tylko odczyt \\ \hline
\texttt{/dashboard/tokens} & Tokeny RFID & CRUD, przypisywanie użytkownikom & CR, aktywacja/dezaktywacja \\ \hline
\texttt{/dashboard/access} & Kontrola dostępu & \multicolumn{2}{c|}{Zarządzanie uprawnieniami, datami wygaśnięcia} \\ \hline
\texttt{/dashboard/logs} & Logi & Przeglądanie, filtrowanie, eksport CSV & Przeglądanie, filtrowanie \\ \hline
\end{tabular}
\caption{Macierz uprawnień panelu administracyjnego (CRUD: Create/Read/Update/Delete).}
\label{tab:admin_permissions}
\end{table}

Oprogramowanie mikrokontrolera realizuje algorytm odczytu UID, wysłania go do endpointu \texttt{/api/v1/access} i oczekiwania na odpowiedź typu boolean \texttt{access.granted}. Panel webowy umożliwia eksport logów do formatu CSV oraz podgląd statystyk w formie wykresów. System wykorzystuje biblioteki: \texttt{MFRC522} do obsługi czytnika RFID, \texttt{WiFiClientSecure} do komunikacji HTTPS oraz \texttt{SPI} do interfejsu komunikacyjnego z czytnikiem.

\begin{figure}[H]
\centering
\includegraphics[width=1\textwidth]{zdjecia/image9.png}
\caption{Interfejs Dashboard - wizualizacja zasobów systemowych i statystyk dostępu w formie wykresów}
\label{fig:admin_panel}
\end{figure}

\section{Testy i wnioski}

\begin{figure}[H]
\centering
\includegraphics[width=1\textwidth]{zdjecia/image1.png}
\caption{Interfejs System Logs - strona logów dostępu z podziałem na udane, odrzucone i nieznane tokeny}
\label{fig:testing}
\end{figure}

Testy systemu uprawnień potwierdziły poprawne blokowanie dostępu dla użytkowników z wygasłą datą ważności tokena oraz dla osób nieprzypisanych do danego skanera. Interfejs administracyjny poprawnie wyświetla logi w czasie rzeczywistym, a filtrowanie zdarzeń działa płynnie. Wnioski z realizacji wskazują, że połączenie systemów wbudowanych z technologiami cloud (Supabase) znacząco upraszcza zarządzanie rozproszoną infrastrukturą kontroli dostępu. Projekt jest gotowy do wdrożenia w małej skali, a potencjalna rozbudowa mogłaby obejmować integrację z systemami alarmowymi oraz aplikację mobilną.

\section{Kosztorys}
Poniższa tabela przedstawia szacunkowe koszty komponentów użytych do budowy jednego punktu dostępowego.

\begin{table}[!ht]
\centering
\begin{tabular}{|l|c|r|r|}
\hline
\textbf{Element} & \textbf{Ilość} & \textbf{Cena jedn.} & \textbf{Suma} \\ \hline
Seeed Studio XIAO ESP32C3 & 1 szt. & 35,00 zł & 35,00 zł \\ \hline
Czytnik RFID-RC522 + karta/brelok & 1 szt. & 15,00 zł & 15,00 zł \\ \hline
Elektrozamek (Solenoid) 12V & 1 szt. & 25,00 zł & 25,00 zł \\ \hline
Moduł przekaźnika 5V / Tranzystor & 1 szt. & 8,00 zł & 8,00 zł \\ \hline
Zasilacz 12V DC / Przetwornica & 1 szt. & 20,00 zł & 20,00 zł \\ \hline
Obudowa, przewody, drobne elementy & - & 20,00 zł & 20,00 zł \\ \hline
\multicolumn{3}{|l|}{\textbf{Suma całkowita}} & \textbf{123,00 zł} \\ \hline
\end{tabular}
\caption{Szacunkowy kosztorys sprzętowy projektu.}
\end{table}

\section{Załącznik}
W niniejszym załączniku przedstawiono dodatkową dokumentację fotograficzną projektu, która szczegółowo ilustruje interfejsy webowe systemu zarządzania dostępem.

\subsection{Interfejsy webowe systemu}

\begin{figure}[H]
\centering
\includegraphics[width=1\textwidth]{zdjecia/image8.png}
\caption{Interfejs Users - zarządzanie użytkownikami systemu i ich rolami (root, admin, user)}
\label{fig:users_interface}
\end{figure}

\begin{figure}[H]
\centering
\includegraphics[width=1\textwidth]{zdjecia/image2.png}
\caption{Interfejs Tokens - zarządzanie zarejestrowanymi tokenami RFID i ich przypisaniami}
\label{fig:tokens_interface}
\end{figure}

\begin{figure}[H]
\centering
\includegraphics[width=1\textwidth]{zdjecia/image3.png}
\caption{Interfejs Scanners - konfiguracja punktów dostępu i skanerów RFID}
\label{fig:scanners_interface}
\end{figure}

\begin{figure}[H]
\centering
\includegraphics[width=1\textwidth]{zdjecia/image6.png}
\caption{Interfejs Access Control - zarządzanie uprawnieniami użytkowników do poszczególnych skanerów}
\label{fig:access_control_interface}
\end{figure}

\subsection{Kod źródłowy projektu}
Pełny kod źródłowy projektu, w tym oprogramowanie mikrokontrolera oraz interfejs webowy, dostępny jest w repozytorium GitHub: \href{https://github.com/Guliveer/RFID-access-manager}{https://github.com/Guliveer/RFID-access-manager}

\end{document}
